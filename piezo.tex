%% ****** Start of file rsitemplate.tex ****** %
%%
%%   This file has been edited from the original source file.
%%	 The original file is part of the revtex4-1 package indicated below.
%%   Version 4.1 of 9 October 2009.
%%
%
% This is a template for producing documents for use with 
% the REVTEX 4.1 document class and the RSI substyle.
% 
% Copy this file to another name and then work on that file.
% That way, you always have this original template file to use.
\documentclass[aip,rsi,reprint]{revtex4-1} % for checking your page length
%\documentclass[aip,rsi,reprint,graphicx,draft]{revtex4-1} % for checking your page length
%\\documentclass[aip,rsi,preprint,graphicx]{revtex4-1} % for review purposes

\usepackage{siunitx}
\usepackage{amsmath}

\usepackage{graphicx}
%\usepackage{caption}
%\usepackage{subcaption}

%% Work with illustrator *.ai files...
\DeclareGraphicsRule{.ai}{pdf}{.ai}{}

\begin{document}

% Use the \preprint command to place your local institutional report number 
% on the title page in preprint mode.
% Multiple \preprint commands are allowed.
%\preprint{}

\title{An ultra-low noise, high-voltage piezo driver}

% repeat the \author .. \affiliation  etc. as needed
% \email, \thanks, \homepage, \altaffiliation all apply to the current author.
% Explanatory text should go in the []'s, 
% actual e-mail address or url should go in the {}'s for \email and \homepage.
% Please use the appropriate macro for the type of information

% \affiliation command applies to all authors since the last \affiliation command. 
% The \affiliation command should follow the other information.

\author{N.C. Pisenti}
\email[]{npisenti@umd.edu}
\author{B.J. Reschovsky}
\author{D.S. Barker}
\author{A. Restelli}
\author{G.K. Campbell}
%\homepage[]{Your web page}
%\thanks{}
%\altaffiliation{}
\affiliation{Joint Quantum Institute, University of Maryland and National Institute of Standards and Technology}

% Collaboration name, if desired (requires use of superscriptaddress option in \documentclass). 
% \noaffiliation is required (may also be used with the \author command).
%\collaboration{}
%\noaffiliation

\date{\today}

\begin{abstract}
We present an ultra-low noise, high voltage driver suited for use with piezoelectric actuators and other low-current applications.
The architecture leverages a commercially available, small-form-factor integrated circuit (IC) for generating high voltage outputs.
The IC uses a flyback configuration switching regulator to generate up to 250V in our design (but up to 1kV or more with small modification), and a high slew-rate op-amp capacitively coupled to the output compensates for the switching noise.
A low-voltage (\SI{\pm 10}{\volt}), high bandwidth modulation input is capable of summing small voltage corrections onto the output, making the driver well suited for use in closed-loop feedback applications.
\end{abstract}

\pacs{}% insert suggested PACS numbers in braces on next line

\maketitle %\maketitle must follow title, authors, abstract and \pacs

% Body of paper goes here. Use proper sectioning commands. 
% References should be done using the \cite and \label commands
\section{Introduction}
\label{Sec:Introduction}

Many instrumentation applications in the modern laboratory require agile, low-noise voltage sources that can supply hundreds of volts or more.
Such high-voltage amplifiers are useful for micro-positioning applications, driving piezo-actuated mirrors (e.g., in a scanning Fabry-Per{\`o}t cavity) or diffraction gratings (such as in extended-cavity diode lasers).
Precision high voltage sources can also be used to bias low-noise, high-bandwidth photodiodes [details?], Other applications...? biophysics? medical devices?
Often, these applications are operated in a closed feedback loop, where small voltage changes on top of a large DC voltage are necessary to correct for some disturbance to the system.

Traditionally, laboratory electronics capable of supplying high voltages fall under one of two architectural umbrellas: switching converters, and ``linear'' amplifiers.
DC-DC converters are efficient and can work at very high voltages, but suffer from switching noise and limited control bandwidths.
Linear-type devices are typically constructed from a high-voltage operational amplifier (op-amp), powered either from a high voltage linear regulator or more typically from a secondary switching converter.
While the op-amp provides \SI{100}{\decibel} or more of power-supply noise rejection, a high-voltage op-amp require substantially more power than an equivalent switching circuit.
In either case, the output voltage  $V_{\text{out}}$ is typically given by
\begin{align}
  V_{\text{out}} &= G_p(V_{\text{DC}} + V_{\text{mod}})\, \text{,}
\end{align}
where $G_p$ is the piezo amplifier gain (typically {$G_p\approx \num{10} - \num{20}$} or more), $V_{\text{DC}}$ is a DC setpoint, and $V_{\text{mod}}$ is a modulation input used for closed-loop control.

It is often the case, however, that in closed-loop feedback applications we do not need and often do not want the modulation control to have a high gain.
Because the closed loop gain cannot be increased without limit, $G_p V_{\text{mod}} \gg 1$ means the servo gain $G$ must be proportionally smaller to keep the lock stable.
Because noise added by the servo is suppressed by $1/|G|$ in the large gain limit, if $G$ is limited to achieve a stable lock it is no longer as effective at servoing its own noise.

To avoid this problem, we desire a high voltage amplifier which separates the ``setpoint'' gain $G_p$ from the modulation gain, $G_m$. That is,
\begin{align}
  V_{\text{out}} &= G_p V_{\text{DC}} + G_m V_{\text{mod}}\, \text{,}
  \label{Eq:PiezoTransfer}
\end{align}
where now we can make $G_m \approx 1$.

\section{Circuit Design}
\label{Sec:Circuit}

\begin{figure}[ht]
\includegraphics[width=\linewidth]{fig/PiezoCircuit.ai}
%\captionsetup{justification=justified}
\caption{Schematic of the high voltage stabilization.
The voltage HV is generated using a Texas Instruments DRV2700 high voltage driver in flyback configuration (see Fig.~\ref{Fig:DRV2700}).
A fast, very high slew-rate op-amp senses the output voltage across $R_1$ and $R_2$, and servos it by modulating the node at ``HV floating gnd''.
The $V_{\text{DC}}$ gain is set by $\left(1+R_1/R_2\right)$, while the modulation gain is set by $-R_{\text{mod}}/R_{\text{fb}}$.
The capacitor linking the floating ground node to the output allows the op-amp to remove residual switching noise and stabilize the DC output according to the transfer function given in Eq.~(\ref{Eq:PiezoTransfer}). \label{Fig:PiezoCircuit}}
\end{figure}

Discussion of individual circuit components


\section{Results}
\label{Sec:Results}
Noise analysis, bandwidth, (DC) stability, etc.


\section{Conclusion}
\label{Sec:Conclusion}


% If in two-column mode, this environment will change to single-column format so that long equations can be displayed. 
% Use only when necessary.
%\begin{widetext}
%$$\mbox{put long equation here}$$
%\end{widetext}

% Figures should be put into the text as floats. 
% Use the graphics or graphicx packages (distributed with LaTeX2e). EPSFig is no longer fully supported.
% See the LaTeX Graphics Companion by Michel Goosens, Sebastian Rahtz, and Frank Mittelbach for examples. 
%
% Here is an example of the general form of a figure:
% Fill in the caption in the braces of the \caption{} command. 
% Put the label that you will use with \ref{} command in the braces of the \label{} command.
%
% \begin{figure}
% \includegraphics{}% % Important NOTE: Please make certain your figures do not include local directory paths. ex. "c:\file\sub\fig1.eps"
% \caption{\label{}}%
% \end{figure}

% Tables may be be put in the text as floats.
% Here is an example of the general form of a table:
% Fill in the caption in the braces of the \caption{} command. Put the label
% that you will use with \ref{} command in the braces of the \label{} command.
% Insert the column specifiers (l, r, c, d, etc.) in the empty braces of the
% \begin{tabular}{} command.
%
% \begin{table}
% \caption{\label{} }
% \begin{tabular}{}
% \end{tabular}
% \end{table}

% If you have acknowledgments, this puts in the proper section head.
%\begin{acknowledgments}
% Put your acknowledgments here.
%\end{acknowledgments}

% Create the reference section using BibTeX:
%\bibliography{your-bib-file}
% Run this once to generate your BBL file. Then copy the contents of your BBL file into your main latex file, commenting out "\bibliography"

\end{document}
%
% ****** End of file aiptemplate.tex ******
